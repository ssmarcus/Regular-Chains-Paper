\usepackage{rotating,graphics,psfrag,epsfig,amssymb,amsmath, subfigure, amsxtra, amsthm, color}
\usepackage{fancyvrb}
\oddsidemargin=-0.25in
\topmargin= -0.5in
\textwidth=7.0in
\textheight=9in

\usepackage{listings}
\lstdefinelanguage{maple}
	{morekeywords={true, false, try, catch, return, break, error, 
	               module, export, local, option, in, use,
                 and, or, not, xor, xnor,
                 if, then, elif, else, fi,
                 while, for, from, by, to, do, od,
                 proc, nargs, local, global, end, NULL}}
\lstset{language=maple,numbers=left,basicstyle=\footnotesize,numberstyle=\tiny, breakatwhitespace=false,frame=single,morecomment=[l]{\#},stringstyle=\ttfamily}

\usepackage[usenames,dvipsnames]{xcolor}
\definecolor{MapleRed}{rgb}{1,0,0}
\definecolor{MapleBlue}{rgb}{0,0,1}
\definecolor{MaplePink}{rgb}{1,0,1}

%\definecolor{MapleRed}{rgb}{0,0,0}
%\definecolor{MapleBlue}{rgb}{0,0,0}

\renewcommand{\arraystretch}{1.25}

%\pagestyle{headings}

%\sloppy

\theoremstyle{plain}
\newtheorem{theorem}{Theorem}
\newtheorem{corollary}{Corollary}
\newtheorem{lemma}{Lemma}
\newtheorem{proposition}{Proposition}
\newtheorem{conjecture}{Conjecture}

\theoremstyle{definition}
\newtheorem{definition}{Definition}
\newtheorem*{notation}{Notation}
\newtheorem{example}{Example}
\newtheorem{remark}{Remark}
\newtheorem{question}{Question}
\newtheorem{problem}{Problem}

\newcommand{\st}{\, \big{|} \,}

\newcommand{\union}{\cup}
\newcommand{\intrsc}{\cap}

\newcommand{\brac}[1]{\left( #1 \right)}
\newcommand{\ideal}[1]{\left< #1 \right>}
\newcommand{\cbrac}[1]{\left\{ #1 \right\}}
\newcommand{\sbrac}[1]{\left[ #1 \right]}

\newcommand{\mvar}{\textbf{mvar}}
\newcommand{\lcoeff}{\textbf{lcoeff}}
\newcommand{\init}{\textbf{init}}
\newcommand{\Ass}{\textbf{Ass}}
\renewcommand{\deg}{\textbf{deg}}

\renewcommand{\implies}{\Rightarrow}

\newcommand{\V}{\mathbf{V}}

\newcommand{\W}{\mathbf{W}}
\newcommand{\M}{\mathbf{M}}
\newcommand{\A}{\mathbf{A}}
\newcommand{\B}{\mathbf{B}}
\newcommand{\C}{\mathbf{C}}
\newcommand{\X}{\mathbf{X}}
\newcommand{\0}{\mathbf{0}}
\newcommand{\II}{\mathbf{I}}
\newcommand{\ZZ}{\mathbb{Z}}
\newcommand{\QQ}{\mathbb{Q}}
\newcommand{\FF}{\mathbf{F}}
\newcommand{\JJ}{\mathbf{J}}
\newcommand{\XX}{\mathbf{X}}
\newcommand{\PP}{\mathbf{P}}


\newcommand{\pdiff}[2]{\frac{\partial #1}{\partial #2}}


\renewcommand{\mod}{\text{ mod }}
\newcommand{\Mod}{\hspace{2mm}\mod}

\newcommand{\rem}{\text{ rem }}
\newcommand{\quo}{\text{ quo }}
\renewcommand{\div}{\text{ div }}
\newcommand{\MM}{\textsf{M}}




