\documentclass[12pt]{article}

\newcommand{\RR}{\mathcal{R}}
\newcommand{\FFT}{\text{FFT}}
\newcommand{\TFT}{\text{TFT}}
\newcommand{\w}{\omega}
\renewcommand{\a}{\mathbf{a}}
\newcommand{\xs}[1]{\mathbf{x_{#1}}}
\newcommand{\kk}{\mathbf{k}}
\newcommand{\Emph}[1]{\textsc{#1}}
\usepackage{algorithm}
\usepackage{algorithmic}
\usepackage{subfigure}
\usepackage{comment}


\usepackage{rotating,graphics,psfrag,epsfig,amssymb,amsmath, subfigure, amsxtra, amsthm, color}
\usepackage{fancyvrb}
\oddsidemargin=-0.25in
\topmargin= -0.5in
\textwidth=7.0in
\textheight=9in

\usepackage{listings}
\lstdefinelanguage{maple}
	{morekeywords={true, false, try, catch, return, break, error, 
	               module, export, local, option, in, use,
                 and, or, not, xor, xnor,
                 if, then, elif, else, fi,
                 while, for, from, by, to, do, od,
                 proc, nargs, local, global, end, NULL}}
\lstset{language=maple,numbers=left,basicstyle=\footnotesize,numberstyle=\tiny, breakatwhitespace=false,frame=single,morecomment=[l]{\#},stringstyle=\ttfamily}

\usepackage[usenames,dvipsnames]{xcolor}
\definecolor{MapleRed}{rgb}{1,0,0}
\definecolor{MapleBlue}{rgb}{0,0,1}
\definecolor{MaplePink}{rgb}{1,0,1}

%\definecolor{MapleRed}{rgb}{0,0,0}
%\definecolor{MapleBlue}{rgb}{0,0,0}

\renewcommand{\arraystretch}{1.25}

%\pagestyle{headings}

%\sloppy

\theoremstyle{plain}
\newtheorem{theorem}{Theorem}
\newtheorem{corollary}{Corollary}
\newtheorem{lemma}{Lemma}
\newtheorem{proposition}{Proposition}
\newtheorem{conjecture}{Conjecture}

\theoremstyle{definition}
\newtheorem{definition}{Definition}
\newtheorem*{notation}{Notation}
\newtheorem{example}{Example}
\newtheorem{remark}{Remark}
\newtheorem{question}{Question}
\newtheorem{problem}{Problem}

\newcommand{\st}{\, \big{|} \,}

\newcommand{\union}{\cup}
\newcommand{\intrsc}{\cap}

\newcommand{\brac}[1]{\left( #1 \right)}
\newcommand{\ideal}[1]{\left< #1 \right>}
\newcommand{\cbrac}[1]{\left\{ #1 \right\}}
\newcommand{\sbrac}[1]{\left[ #1 \right]}

\newcommand{\mvar}{\textbf{mvar}}
\newcommand{\lcoeff}{\textbf{lcoeff}}
\newcommand{\init}{\textbf{init}}
\newcommand{\Ass}{\textbf{Ass}}
\renewcommand{\deg}{\textbf{deg}}

\renewcommand{\implies}{\Rightarrow}

\newcommand{\V}{\mathbf{V}}

\newcommand{\W}{\mathbf{W}}
\newcommand{\M}{\mathbf{M}}
\newcommand{\A}{\mathbf{A}}
\newcommand{\B}{\mathbf{B}}
\newcommand{\C}{\mathbf{C}}
\newcommand{\X}{\mathbf{X}}
\newcommand{\0}{\mathbf{0}}
\newcommand{\II}{\mathbf{I}}
\newcommand{\ZZ}{\mathbb{Z}}
\newcommand{\QQ}{\mathbb{Q}}
\newcommand{\FF}{\mathbf{F}}
\newcommand{\JJ}{\mathbf{J}}
\newcommand{\XX}{\mathbf{X}}
\newcommand{\PP}{\mathbf{P}}


\newcommand{\pdiff}[2]{\frac{\partial #1}{\partial #2}}


\renewcommand{\mod}{\text{ mod }}
\newcommand{\Mod}{\hspace{2mm}\mod}

\newcommand{\rem}{\text{ rem }}
\newcommand{\quo}{\text{ quo }}
\renewcommand{\div}{\text{ div }}
\newcommand{\MM}{\textsf{M}}




	
\pagestyle{myheadings}

\title{Regular Chains (Theory)}
\author{Steffen Marcus, Marc Moreno Maza, \'Eric Schost and Paul Vrbik}
\date{\today}

\begin{document}

\markboth{ppp}{\textbf{Regular Chains (Theory)}}
\maketitle


\begin{abstract}
A self contained survey of the theory of regular chains.
\end{abstract}

\section*{Example Bank}

For sure make a change.

\begin{example}[From \cite{ALM99}]  
In this example, we define the \Emph{iterated initials} of a polynomial $p\in\PP_n$.  These form a triangular set $\textbf{iter}(p)\subset \PP_n$ defined recursively by the rule $\textbf{iter}(p)=\{p\}\bigcup\textbf{iter}(\init(p))$ and $\textbf{iter}(p) = \emptyset$ for $p\in k$.  The polynomial $p=(x_2x_3-1)x_4+x_2^2$ has iterated initials $\textbf{iter}(p)=\{x_2,x_2x_3-1,p\}$.
\end{example}

\begin{example} [From \cite{ALM99}]  
Let $n\geq 3$ and assume $k=\mathbb{Q}$ so that our polynomials are defined over the rational numbers.  Define polynomials
\begin{align*}
p_1 &= x_1^4-5x_1^2+6,\\
p_2 &=(x_1^2-2)x_2^2+(-2x_1^3+4x_1)x_2+x_1^4-2x_1^2,\\
p_3 &=(x_1^2-2)x_3^4+(x_2-x_1)x_3^3+(1-x_1)x_3+1,
\end{align*}
and set $T_1=\{p_1\}$ and $T_2=\{p_1,p_2\}$.  These are both triangular sets, but $T_2$ is not a regular chain.  Consider the primary decomposition
\[
\left<T_2\right>=\left<x_1^2-3,(x_2-x_1)^2\right>\cap\left<x_1^2-2\right>
\]
with respective associated primes $P_1=\left<x_1^2-3,(x_2-x_1)\right>$ and $P_2=\left<x_1^2-2\right>.$  Notice also that $\left<x_1^2-2\right>$ is an associated prime of $\left<p_1\right>$.  Since $\init(p_2)$ vanishies with respect to this ideal, $T_2$ is not a regular chain.  It is straightforward to see how $T_2$ can be encoded by regular chains, since both $\{x_1^2-3,(x_2-x_1)^2\}$ and $\{x_1^2-2\}$ are regular chains and $P_1$ and $P_2$ are the respective radicals of their saturation ideals.  Finally, notice that $T_3=\{x_1^2-3,(x_2-x_1)^2, p_3\}$ is also a regular chain, and $\sqrt{\textbf{Sat}(T_3)}=\left<x_1^2-3,x_2-x_1, (x_3^2-x_1)(x_3^2+1)\right>$.
\end{example}

\begin{example}[From \cite{LMPX08}]  
Let $n=4$ and take $T=\{x_1x_3+x_2,x_2x_4+x_1\}$.  Then 
\[
\left<T\right>=\left<x_1,x_2\right>\cap\left<x_1x_3+x_2,-x_3x_4+1\right>,
\]
whereas
\[
\textbf{Sat}(T)=\left<x_1x_3+x_2,-x_3x_4+1\right>,
\]
This is an example where $\textbf{Sat}(T)$ is strictly larger than $\left<T\right>$.
\end{example}

\begin{example}[From \cite{LMPX08}]  
Let $n=3$ and set $T_1 = \{x_1x_2\}$ and $T_2 = \{x_1x_2,x_1x_3\}$.  Then $T_1$ and $T_2$ are regular chains with $\textbf{Sat}(T_1)=\left<x_2\right>$ and $\textbf{Sat}(T_2) = \left<x_2,x_3\right>$.
\end{example}

\section*{Commutative Algebra}

{\color{red} I am following \cite{V98} for most, if not all, of this.  It's a text I am learning to love btw.}

\begin{notation}
Denote by $A$ an arbitrary commutative Noetherian ring.  We will usually denote ideals in $A$ by $I, J, \ldots$ which, since $A$ is Noetherian, are finitely generated.
\end{notation}

\begin{definition}[ideal quotient]
Let $I, J$ be ideals in $A$.  The \Emph{ideal quotient} of $I$ by $J$ is the ideal
\[
I:J=\{a \in A : aJ\subset I\}.
\]
\end{definition}

\begin{definition}
An element $f\in A$ is regular modulo $I$ if its canonical image in the quotient $A/I$ is a regular element (meaning a nonzerodivisor).
\end{definition}

%\begin{remark}
%The ideal quotient gets its name because, for an ideal $K$, $IJ\subset K$
%\end{remark}

\begin{definition}[saturation]
Let $I, J$ be ideals in $A$.  The \Emph{saturation} of $I$ with respect to $J$ is the ideal
\[
I:J^\infty=\bigcup_{n \geq 1}(I:J^n).
\]
\end{definition}

\begin{remark}
The ideal quotient gets its name because, for K an ideal, $IJ\subset K \Leftrightarrow I\subset (K:J)$.  The saturation ideal gets its name from the notion of saturation for the localization of rings by multiplicative sets.  That is, for $S$ a multiplicative set of $A$, the localization of $I$ at $S$ is the set 
\[
_{S}I=\{a\in A| \exists s\in S, sa\in I\}. 
\]
and this set is sometimes also called the saturation of the ideal $I$ with respect to $S$.  When $S$ is finitely generated by elements $s_1,\ldots,s_n$, and we set $f=\prod s_i$, then $_SI=I:f^\infty$.
\end{remark}

\begin{proposition}
The element $f\in A$ is regular modulo an ideal $I$ if and only if 
\[
I:f = I \text{ or } I:f^\infty = I.
\]
\end{proposition}
\begin{proof}
Let $\bar f$ be the canonical image of $f$ in $\A/I$.  We first assume $I:f = I \text{ or } I:f^\infty = I$. If $I:f=I$, then the elements $a\in A$ such that $af\in I$ are precisely the elements of $I$.  Take $a\in A$ such that $\overline{af}=\bar0$ in $A/I$.  This is equivalent to $af \in I$ in $A$, so this part of this direction is immediate.  If we don't have $I:f=I$, we instead have $I:f^\infty=I$.  Again take $a\in A$ so that $\overline{af}=\overline{0}$.  Then $af\in I$, and we need only multiply by a sufficiently high enough power of $f$ to get $a\in I$.  For the other direction, assume $\bar f$ is not a zero-divisor.  Then $af^n\in I$ implies $\overline{af^n}=\bar0$, which in turn implies $\overline{af^{n-1}}=\bar0$.  Continue reducing the exponent of $\bar f$ until we get $a\in I$.
\end{proof}



\begin{proposition}
The ideal $I:J^\infty$ removes from any primary decomposition of $I$ all $P_i$-primary components for whom $J$ intersects $P_i$.  In other words, $I:J^\infty$ is the intersection of the primary components for whom $P_i\cap J=\emptyset$.
\end{proposition}
\begin{proof}
Given any primary ideal $Q$ in $A$, $Q:J^\infty=Q$ if $J\cap\sqrt{Q}=\emptyset$ and $Q:J^\infty=A$ if $J\cap\sqrt{Q}\neq\emptyset$. (expand on this?)
\end{proof}

\begin{definition}[annihilator]
For a nonzero $m\in A/I$, define the \Emph{annihilator} of $m$ to be the set
\[
0:m=\{a\in A| \overline am=0\}.
\]
Similarly, define the annihilator of $B\subset A/I$ to be the set $0:B=\cbrac{a\in A \st \overline aB=0}$.
\end{definition}

\begin{definition}[associated prime]
We say that a prime ideal $P$ in $A$ is an \Emph{associated prime} of the ideal $I$ (or simply, $P \in \Ass_A(I)$) when there is an $m \in A/I$ such that $P = 0:m$. That is,
\[
P \in \Ass_A(I) \iff \exists m \in A/I \text{ s.t. } P = \cbrac{a \in A \st \overline am=0} .
\]
\end{definition}

\begin{definition}[primary]
An ideal $I$ is said to be \Emph{primary} if $\Ass_A(I)={P}$ contains only one prime ideal.  We may refer to $I$ as $P$-primary in order to differentiate between primary ideals with different associated primes.  
\end{definition}

\begin{remark}
Since $A$ is noetherian, $\Ass_A(I)$ will always be a finite set.  The minimal primes of $I$ are a subset of the associated primes, and those that are not minimal are called embedded primes of $I$.  In other words, an ideal $P\in \Ass_A(I)$ is an embedded prime of $I$ if it properly contains another element of $\Ass_A(I)$.  Also, the following gives an interesting connection between associated ideals and quotient ideals: 
\[
P\in\Ass_A(I) \Leftrightarrow I:(I:P)\subset P.
\]
The intersection of two $P$-primary ideals is again $P$-primary.
\end{remark}

\begin{definition}[irreducible ideal]
An ideal $I$ is \Emph{irreducible} if it is not the intersection of two other ideals in which it is properly contained.  
\end{definition}

\begin{proposition}[]
\end{proposition}

\begin{theorem}[Lasker-Noether theorem \cite{BLM11}, \cite{V98}, Theorem 3.1.1]
Every ideal is a finite intersection of irreducible ideals.  Every irreducible ideal is primary.  Thus any ideal $I$ can be represented as a finite intersection of primary ideals
\[
 I=J_1\cap\cdots \cap J_n.
 \]
Such a representation is not necessarily unique, but can be minimized by removing redundant primary ideals and intersecting $P$-primary ideals for each associated prime $P$.  This forms a minimal primary decomposition with a uniquely defined number of components, one for each associated prime ideal.
\end{theorem}

\begin{definition}[dimension]
We define the \emph{dimension} of $A$ to be the supremum of the lengths of chains of prime ideals in $A$.  The dimension of $I$ is then defined to be the dimension of the quotient ring $R/I$.  When $I$ is prime, we can localize away from $I$ and define the codimension of $I$ to be dimension of the local ring $A_I$ (when $I$ is not prime, just take the minimum codimension over all primes containing $I$).  
\end{definition}

\begin{definition}[unmixed and equidimensional]
We call $I$ \emph{equidimensional} if all its minimal primes have equal codimension.  Further, we call $I$ \emph{unmixed dimensional} if it is equidimensional and has no embedded primes, that is, all associated primes are the same codimension. 
\end{definition}

\begin{definition}
A \Emph{regular sequence} on $A$ is a sequence of elements $a_1,\ldots,a_r\in A$ such that the ideal $\left<a_1,\ldots,a_r\right>$ is proper and for each $i$ the canonical image of $a_{i+1}$ in $A/\left<a_1,\ldots,a_i\right>$ is a nonzerodivisor. 
\end{definition}

\begin{remark}
Assume $I$ is an ideal of codimension $m$.  What we would like is not only a minimal primary decomposition $I=J_1\cap\cdots\cap J_n$, but even more, such a composition grouped into equidimensional collections of the primary ideals.  Implicitly, this is not complicated.  Given any minimal primary decomposition, intersect those primary components of the same codimension, producing equidimensional ideals $I_i$ of codimension $i$.  In general, we can produce a representation 
\[
I=I_m\cap\cdots\cap I_{\text{dim}(A)}.
\]   
Using some (rather complicated) homological algebra and local cohomology, this can be accomplished explicitly rather than implicitly in any Gorenstein ring $A$ by reversing the process and first computing a decomposition into equidimensional components $l_i$.  The process is as follows:
\begin{enumerate}
\item Decompose $I$ into equidimensional ideals $I=I_m\cap\cdots\cap I_{\text{dim}(A)}$ whose associated primes are also associated primes of $I$.\\
\item Decompose the corresponding radicals into minimal primary decompositions 
\[
\sqrt{I_i} = \bigcap_j J_{i,j}.
\]\\
\item Set $K_{i,j}=I_i:(I_i:J_{i,j}^\infty)$.  Then the desired minimal primary decomposition is 
\[
I=\bigcap_{i,j}K_{i,j}.
\]
\end{enumerate}
The major difficulty is explicitly computing the $I_i$ in step 1.  This is solved by using regular systems, in particular, by the following criterion for unmixedness:

\begin{proposition}[\cite{V98} Corollary 3.2.2]
Assume $A$ is Gorenstein and $I$ is codimension $m$, with ${\bf{x}}=\cbrac{x_1,\ldots,x_m}\subset I$.  Then $I$ is unmixed dimensional if and only if $I=\left<\bf{x}\right>:( \left<\bf{x}\right>:I)$.
\end{proposition}

\end{remark}
%-------------------------------------------------------------
\begin{comment}
\section*{Preliminary Questions and Ideas of Steffen}
\begin{remark}So far I have been reading the following five papers: \cite{ALM99,AM99,H03,LMPX08,M00}.  Put together they give a nice overview of the theory.  At this first glance, I'm having trouble understanding exactly why the theory necessitates a survey paper like the one we are proposing, although I do see that the exposition so far is a bit disjointed.  I think our reader will be best served by a paper that includes a TON of examples (it seems, Paul, you agrees with this given the way you've started writing), and motivates not only the theory for theory's sake, but how the reader may be able to use regular chains to better their research (and life).
\end{remark}

\begin{remark} A first attempt at an outline might be something like the following:
\begin{enumerate}
\item Introduction - statement of questions being addressed, one or two examples, a brief history of the theory
\item First examples - one or two long drawn out examples describing some interesting varieties using both Kalkbrenner and Larard decompositions and explaining the intricacies as the examples go.
\item Commutative algebra - go through necessary pre-req's
\item Regular Chains - go through theory of triangular sets and regular chains.  
\item Maybe a section on the notion of a Primitive regular chain
\item Triangular Decompositions
\item Other applications
\end{enumerate}
\end{remark}
\end{comment}
%-------------------------------------------------------------


\section{Introduction}
We start by asking a rudimentary question:

\begin{question} What does it mean to solve a system of polynomials? To be more precise, let $\kk$ be a field, and denote by $K$ it's algebraic closure.  Given a subset $F=\cbrac{f_0, \ldots, f_m}$ of $\kk[x_1,\ldots,x_n]$, what does it mean to ``usefully'' encode the set of points in $K^n$ where the $f_i$'s vanish simultaneously? 
\end{question}

Perhaps, as in algebraic geometry, one would be satisfied to decompose the variety 
$$\V(F) = \cbrac{ p \in K^n \st f_i(p)=0, \,\forall f_i \in F}$$
into its irreducible components. However (in addition to being computationally infeasible \cite{}), this decomposition may not help with the actual construction of points in the affine space.

Switching to the algebraic approach from the geometric. We may try to manipulate $\ideal{F}$ directly, by say, finding its Gr\"obner basis with respect to the lexical ordering. Elimination theory ensures that we get another, equivalent ideal $\ideal{G} = \ideal{F}$ with a crucial property. The generators of $\ideal{g_1, \ldots, g_n} = \ideal{G}$ have mutually different largest variable (e.g. $g_1 \in \kk[x_1], g_2 \in \kk[x_1,x_2], \ldots)$ meaning (in the ``nicest'' zero-dimensional case) that we can solve for $x_1$, back substitute and solve for $x_2$, and repeat for all $x_i$. Of course complications arise when the solutions are of higher dimension; but surely this is an attractive quality for our ``useful encoding''.

\section{First Examples}

Check out the example on the second page of \cite{AM99}.  That is a really nice example, and sort of a baby of the type of examples I am thinking about for this section.

\section{Algebraic Preliminaries}

\begin{remark} among all the obvious things for this section, I am particularly interested in building a clear understanding of the equidimensionality result of the theory.  In particular, we should be able to explain how saturations are unmixed dimensional (not just equidimensional), and what that ends up meaning.  We should also try to build an algebraic understanding of Mark's definition of `strongly equidimensional', which is something I still haven't found in the literature.
\end{remark}

\section{Regular Chains}


\subsection{Triangular Sets}

Let $\kk$ be a field, and $x_n \succ x_{n-1} \succ \cdots \succ x_1$ be $n$ ordered variables over the monomial ordering $\succ$. For each $i=1,\ldots,n$, define $\PP_i=\kk[x_1,\ldots,x_n]$ to be the ring of multivariate polynomials in the variables $x_1, \ldots, x_n$ with coefficients in $\kk$ (we also set $\PP_0 = \kk$). Let $p\in \PP_n$ be a non-constant polynomial.

Let $\mvar(p)$ denote the $\succ$-largest $x_i$ such that $\deg(p,x_i)>0$ (which, for the purposes of this paper, will always correspond to the $>$-largest $i$).


Let us call such a set of polynomials with mutually different largest variable a \textsc{triangular set}.


\begin{definition}[Triangular-Set]
A subset $T$ of $\PP_n$ is a \textsc{triangular set} if there is no $t \in T$ such that $t \in \kk$. And, for all pairs, $p,q \in T$ with $p \neq q$ we have $\mvar(p) \neq \mvar(q)$.
\end{definition}



\begin{example} $T = \cbrac{x_1-x-1^2,x_2^2 - x_1, x_1x_3^2 - 2x_2x_3+1, (x_2x_3-1)x_4+x_2^2} \subset \PP_4$ is a triangular set because
\begin{align*}
(x_2x_3-1)x_4+x_2^2		&\in \kk[x_1,x_2,x_3,x_4] \\
x_1x_3^2 - 2x_2x_3+1		&\in \kk[x_1,x_2,x_3] \\
x_2^2 - x_1				&\in \kk[x_1,x_2] \\
x_1-x-1^2				&\in \kk[x_1]
\end{align*}
(the triangular shape of the polynomial rings as they are written above was the inspiration for the name ``triangular'' set).
\end{example}

So, addressing the original question. Suppose that we can take any variety $\V(F)$ and find \emph{some} (i.e. non-unique) set of triangular sets $\cbrac{T_1,\ldots,T_k}$ such that $\V(T_1) \cup \cdots \cup \V(T_k) = \V(F)$. This a good start but is still (quite) incomplete. Consider the following example.

\begin{example}\label{Example::BadBackSubstition}
Example where back substitution breaks everything.
\end{example}

\noindent Clearly, we need some additional restrictions to get the ``nice'' back substitution we desire (see \S \ref{Section::BackSub}).

\subsection{Existence of Triangular Decompositions}
Before we try to impose special behaviour of our triangular decompositions it is prudent to investigate that they actually exist for all arbitrary (finite) subsets of $\PP_n$. 

\begin{theorem}
For any finite subset $F$ of $\PP_n$ there exists triangular sets $T_1, \ldots, T_n$ such that 
$$\V(F) = \V(T_1) \cup \cdots \cup \V(T_n).$$
\end{theorem}
\begin{proof}
There is some (super complicated) constructional proof by Wu, or by Changbo, or by MMM in MEGA.
\end{proof}

\subsection{Nice Properties of Triangular Sets}
%not actually part of paper
\noindent\textsc{Informal breakdown}
\begin{enumerate}
\item Each $T_i$ encodes a (strongly?) equidimensional subset of $\V(F)$ and all the components of $\V(F)$ get encoded (trivial by existence).
\item the set of things that reduce to zero modulo $T$ is $\cbrac{p \in \PP_N \st \text{prem}(p,T)=0}$ and is also a subset of $\text{sat}(T)$.
\item Euclids algorithm in $n$-dimensions
\end{enumerate}
%%%%


\subsection{``Well-behaved'' Back Substitution}\label{Section::BackSub}
%not actually part of paper
\noindent\textsc{Informal breakdown}
\begin{enumerate}
\item Initially reduced triangular sets make sure leading coefficients don't vanish
	\begin{enumerate}
		\item need pseudo-remainders
	\end{enumerate}
\item $\sqrt{\ideal{T}}:h = \sqrt{\text{sat}(T)}$, $h$ is the pr
\item The ``regular zeros of T'' ($\V(T) - \V(h)$).
\end{enumerate}


The problem with Example \ref{Example::BacBackSubstitution} is that the ``leading coefficient'' of $f_?$ vanishes. This section introduces the necessary restrictions on triangular sets to eliminate this ``degenerate'' case, towards our ultimate goal of ``usefully'' encoding the solution space of an arbitrary $F$.

\begin{definition}[Initial]
For a polynomial $p \in \kk[x_1,\ldots,x_m]$ with $\mvar(p)=x_m$, we denote $\init(p)$ to be the leading coefficient (in the usual sense) of $p$ when it is regarded as a univariate polynomial from $\brac{\kk[x_1,\ldots,x_{m-1}]}[x_m]$ (i.e. a polynomial in $x_m$ with coefficients from $\kk[x_1,\ldots,x_{m-1}]$).
\end{definition}

\begin{definition}[Regular Zero]
A point $a\in\V(T)$ is called a \textsc{regular zero} of $T$ if for every $p\in T$ we have that $\init(p)$ does not vanish when evaluated at $a$.  Denote by $\W(T)$ the set of regular zeroes.  
\end{definition}

\begin{example} Of an initial.
\end{example}

We can guarantee an initial will not vanish by moving to some residue ring where that initial is regular (i.e. not a zero divisor). Alternatively, this condition is equivalent to ensuring our polynomials are monic in some special field of fractions---but, more on this later. First, recall the following definitions.

{\color{red}
\begin{remark}
I think the best idea of what we mean by well-behaved back substitution is as it is described in \cite{H03}.  That is, for any algebraic variable $x_i$, the extension of generic zeros from $T_{x_i}^{-}$ to $T_{x_{i+1}}^{-}$.  In particular, the way Hubert describes it, the definition of a regular chain gives us exactly the property that all associated primes of $\textbf{Sat}(T_{x_{i+1}}^{-})$ are given exactly as the ``cutbacks" of the associated primes of $\textbf{Sat}(T)$.  Said better, $\text{Ass}(\textbf{Sat}(T_{x_{i+1}}^{-}) = \{P\cap\PP_{i}|P\in\text{Ass}(\textbf{Sat}(T))\}$.  The important statements are Proposition 5.8 and examples 4.6 and 5.9 in Hubert's paper. 
\end{remark}}
\subsection{Regular Chains}

\section{Primitive Regular Chains}

\section{Decompositions}

\subsection{On Kalkbrenner Decompositions}
\subsection{On Lazard Decompositions}

\section{Other Applications}
\bibliographystyle{amsalpha}             % (uses file "plain.bst")
\bibliography{regchains}       % expects file "myrefs.bib"
\end{document}